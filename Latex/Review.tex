\documentclass[12pt,a4paper]{article}
\usepackage[utf8]{inputenc}
\usepackage[T1]{fontenc}
\usepackage{amsmath}
\usepackage{amsfonts}
\usepackage{amssymb}
\usepackage{makeidx}
\usepackage{graphicx}
\author{Group 9}
\begin{document}
	\section{Basic Concepts of Time Series Modeling}
	\subsection{Definition of A Time Series}
	A time series is a sequential set of data points, measured typically over successive times. It is mathematically defined as a set of vectors $$x_{(t)},t = 0,1,2,... $$where t represents the time elapsed [21, 23, 31]. The variable ) x(t is treated as a random variable. The measurements taken during an event in a time series are arranged in a proper chronological order. A time series containing records of a single variable is termed as univariate. But if records of more than one variable are considered, it is termed as multivariate. A time series can be continuous or discrete. In a continuous time series observations are measured at every instance
	of time, whereas a discrete time series contains observations measured at discrete points of time. For example temperature readings, flow of a river, concentration of a chemical process etc. can be recorded as a continuous time series. On the other hand population of a particular city, production of a company, exchange rates between two different currencies may represent
	discrete time series. Usually in a discrete time series the consecutive observations are recorded at equally spaced time intervals such as hourly, daily, weekly, monthly or yearly time separations. As mentioned in [23], the variable being observed in a discrete time series is assumed to be measured as a continuous variable using the real number scale. Furthermore a continuous time series can be easily transformed to a discrete one by merging data together over a specified time interval.\
	
	\subsection{ Components of a Time Series}
	
	A time series in general is supposed to be affected by four main components, which can be separated from the observed data. These components are: Trend, Cyclical, Seasonal and Irregular components. A brief description of these four components is given here. The general tendency of a time series to increase, decrease or stagnate over a long period of time is termed as Secular Trend or simply Trend. Thus, it can be said that trend is a long term movement in a time series.\\
	Seasonal variations in a time series are fluctuations within a year during the season. The important factors causing seasonal variations are: climate and weather conditions, customs, traditional habits, etc. For example sales of ice-cream increase in summer, sales of woolen cloths increase in winter. Seasonal variation is an important factor for businessmen, shopkeeper and producers for making proper future plans.The cyclical variation in a time series describes the medium-term changes in the series, caused by circumstances, which repeat in cycles. The duration of a cycle extends over longer period of time, usually two or more years. Most of the economic and financial time series show some kind of cyclical variation. \\
	There are two different types of models are generally used for a time series.\
	
	Multiplicative Model: $$ Y(t) = T(t)× S(t)×C(t)× I(t)$$ \
	Additive Model: $$Y(t) = T(t) + S(t) + C(t) + I(t)$$\
	
	Here Y(t) is the observation and  T(t), S(t), C(t)and I(t) are respectively the trend, seasonal,
	cyclical and irregular variation at time $$t$$.\
	Multiplicative model is based on the assumption that the four components of a time series are
	not necessarily independent and they can affect one another; whereas in the additive model it is
	assumed that the four components are independent of each other.
	\newpage
	\chapter{Review}
	Time series modeling is a dynamic research area which has attracted attentions of researchers community over last few decades. The main aim of time series modeling is to carefully collect and rigorously study the past observations of a time series to develop an appropriate model which describes the inherent structure of the series. This model is then used to generate future values for the series, i.e. to make forecasts. Time series forecasting thus can be termed as the act of predicting the future by understanding the past [31]. Due to the indispensable importance
	of time series forecasting in numerous practical fields such as business, economics, finance, science and engineering, etc. [7, 8, 10], proper care should be taken to fit an adequate model to the underlying time series. It is obvious that a successful time series forecasting depends on an appropriate model fitting. A lot of efforts have been done by researchers over many years for the development of efficient models to improve the forecasting accuracy. As a result, various important time series forecasting models have been evolved in literature.\
	
	One of the most popular and frequently used stochastic time series models is the
	Autoregressive Integrated Moving Average (ARIMA) [6, 8, 21, 23] model. The basic
	assumption made to implement this model is that the considered time series is linear and follows a particular known statistical distribution, such as the normal distribution. ARIMA model has subclasses of other models, such as the Autoregressive (AR) [6, 12, 23], Moving Average (MA) [6, 23] and Autoregressive Moving Average (ARMA) [6, 21, 23] models. For seasonal time series forecasting, Box and Jenkins [6] had proposed a quite successful variation of ARIMA model, viz. the Seasonal ARIMA (SARIMA) [3, 6, 23]. The popularity of the ARIMA model is mainly due to its flexibility to represent several varieties of time series with simplicity as well as the associated Box-Jenkins methodology [3, 6, 8, 23] for optimal model building process. But the severe limitation of these models is the pre-assumed linear form of the associated time series which becomes inadequate in many practical situations. To overcome this drawback, various non-linear stochastic models have been proposed in literature
	[7, 8, 28]; however from implementation point of view these are not so straight-forward and simple as the ARIMA models.\
	
	Recently, artificial neural networks (ANNs) have attracted increasing attentions in the  domain of time series forecasting [8, 13, 20]. Although initially biologically inspired, but later on ANNs have been successfully applied in many different areas, especially for forecasting and classification purposes [13, 20]. The excellent feature of ANNs, when applied to time series forecasting problems is their inherent capability of non-linear modeling, without any presumption about the statistical distribution followed by the observations. The appropriate model is adaptively formed based on the given data. Due to this reason, ANNs are data-driven and self-adaptive by nature [5, 8, 20]. During the past few years a substantial amount of research works have been carried out towards the application of neural networks for time series modeling and forecasting. A state-of-the-art discussion about the recent works in neural networks for tine series forecasting has been presented by Zhang et al. in 1998 [5]. There are
	various ANN forecasting models in literature. The most common and popular among them are the multi-layer perceptrons (MLPs), which are characterized by a single hidden layer Feed Forward Network (FNN) [5,8]. Another widely used variation of FNN is the Time Lagged Neural Network (TLNN) [11, 13]. In 2008, C. Hamzacebi [3] had presented a new ANN model, viz. the Seasonal Artificial Neural Network (SANN) model for seasonal time series forecasting.\
	His proposed model is surprisingly simple and also has been experimentally verified to be quite successful and efficient in forecasting seasonal time series. Offcourse, there are many other existing neural network structures in literature due to the continuous ongoing research
	works in this field. However, in the present book we shall mainly concentrate on the above mentioned ANN forecasting models.\
	
	A major breakthrough in the area of time series forecasting occurred with the development of Vapnik’s support vector machine (SVM) concept [18, 24, 30, 31]. Vapnik and his co-workers designed SVM at the AT and T Bell laboratories in 1995 [24, 29, 33]. The initial aim of SVM was to solve pattern classification problems but afterwards they have been widely applied in many other fields such as function estimation, regression, signal processing and time
	series prediction problems [24, 31, 34]. The remarkable characteristic of SVM is that it is not only destined for good classification but also intended for a better generalization of the training data. For this reason the SVM methodology has become one of the well-known techniques, especially for time series forecasting problems in recent years. The objective of SVM is to use the structural risk minimization (SRM) [24, 29, 30] principle to find a decision rule with good generalization capacity. In SVM, the solution to a particular problem only depends upon a subset of the training data points, which are termed as the support vectors [24, 29, 33]. Another important feature of SVM is that here the training is equivalent to solving a linearly constrained quadratic optimization problem. So the solution obtained by applying SVM method is always unique and globally optimal, unlike the other traditional stochastic or neural network methods [24]. Perhaps the most amazing property of SVM is that the quality and complexity of the solution can be independently controlled, irrespective of the dimension of the input space [19, 29]. Usually in SVM applications, the input points are mapped to a high dimensional feature space, with the help of some special functions, known as support vector kernels [18, 29, 34], which often yields good generalization even in high dimensions.During
	the past few years numerous SVM forecasting models have been developed by researchers. In this book, we shall present an overview of the important fundamental concepts of SVM and then discuss about the Least-square SVM (LS-SVM) [19] and Dynamic Least-square SVM (LSSVM) [34] which are two popular SVM models for time series forecasting. The objective of this book is to present a comprehensive discussion about the three widely popular approaches for time series forecasting, viz. the stochastic, neural networks and SVM approaches. This book contains seven chapters, which are organized as follows: Chapter 2
	gives an introduction to the basic concepts of time series modeling, together with some associated ideas such as stationarity, parsimony, etc. Chapter 3 is designed to discuss about the various stochastic time series models. These include the Box-Jenkins or ARIMA models, the generalized ARFIMA models and the SARIMA model for linear time series forecasting as well as some non-linear models such as ARCH, NMA, etc. In Chapter 4 we have described the application of neural networks in time series forecasting, together with two recently developed models, viz. TLNN [11, 13] and SANN [3]. Chapter 5 presents a discussion about the SVM concepts and its usefulness in time series forecasting problems. In this chapter we have also briefly discussed about two newly proposed models, viz. LS-SVM [19] and DLS-SVM [34] which have gained immense popularities in time series forecasting applications. In Chapter 6,
	we have introduced about ten important forecast performance measures, often used in literature, together with their salient features. Chapter 7 presents our experimental forecasting results in terms of five performance measures, obtained on six real time series datasets, together with the associated forecast diagrams. After completion of these seven chapters, we have given a brief conclusion of our work as well as the prospective future aim in this field. 
	
	\newpage
	LULC data are records that documents to what extent a region is covered by wetlands, forests, agriculture, impervious surfaces, and 	other land and water forms. These water forms include open water  or wetlands. Land use shows how people use landscape either for conservation, development, agriculture or mixed uses [6, 7]. Changes In land can be identified by analysing satellite imagery. However, land use cannot be identified from satellite imagery. Satellite imagery give us  information that helps in understanding the present landscape. Furthermore, to see changes	through time, different years are needed. With this information, we can assess  decades of data as well as  insight into the possible effects of these changes that has occured and make better decisions before they can cause great harm.	According to [10], five defferent types of LULC pattern were classified barren lands such as Galamsey Site, agricultural lands, urban lands, quarries, and free water bodies, to detect the 25years LULC change in the western Nile delta of Egypt. Supervised maximum likelihood classification (MLC)	method together with landsat images were used in Erdas Imagine software. The finding shows a significant
	change in barren land changing into agricultural land continuously	from 1984 to 2009.\\
	
	Similarly, in [1] used the maximum likelihood algorithm (MLA)	and Markov chain model (MCA) to study the LULC classification	using ArcGIS and future prediction using Idiri respectively in	Kathmandu city Nepal. Built-up, water body, forest area, open	field and cultivated land classes classified. Results show built-up	area significantly increased, and water body, forest area, open	field and cultivated land decrease downward trend from 1976 to
	2009. Furthermore, the Markov chain Analysis prediction for 2017	shows that in 2017 Urban area will increase to cover 72.24 $\%$ of	the total land in Kathmandu and cultivated land remains only	20.90 $\%$. Waterbody and the open field will increase respectively	by 0.59$\%$, 0.19$\%$ whereas forest land decrease by 0.47$\%$.\\
	
	Furthermore, in [10] used the Maximum likelihood classification	(MLC), Change detection and spatial matrix analysis to analyse	land cover change of fifty-year period (1954 to 2004) in Avellino 	Italy. The result shows 4 LULC classes, with urban land use increasing rapidly affecting the cultivated land mostly, while woodland and grassland cover decrease was at a lower rate.	Moreover, in [11] studied the LULC changes and structure in	Dhaka metropolitan, Bangladesh in a period of 1975 to 2005.	Maximum Likelihood Classification (MLC) and transition matrix	method were used for LULC classification and rate/ pattern of	LULC. The Result shows six classes in LULC of the water body,
	vegetation, bare soil, cultivated land built-up and wetland/lowland.	Also, a significant increase in the built-up land, while cultivated	land, vegetation and wetland decreased accordingly from 1975 to	2005.\\
	
	Also, in [12] used Maximum Likelihood Classification and comparison method to study the LULC classification and change respectively from 1976 to 2003 in Tirupati, India. The results show	6 LULC classes, agricultural land, built-up, dense forest, plantation, water spread and other land, a significant increase in built-up	area, plantation forests and other land, while a decrease on the part	of the waterbody, dense forest and agricultural land was noticed.\\
	
	Moreover, in [13] studied the LULC change in Duzce plain Turkey. Supervised classification and the Corine land cover nomenclature methods used. The result shows 5 LULC classes as urban	fabric, forest, heterogeneous agricultural land, inland wasteland	and (Industrial, commercial, and transport) units with an accuracy	assessment between 92.41 $\%$ and 97.3 $\%$ for LULC map 2010 and
	1987 respectively. Also, a significant change in LULC was noticed with 11.2$\%$ increase in agricultural area and 335$\%$ decrease	of forest land.\\
	

	
	Also, a significant increase and a decrease of LULC were noticed between the years 1973, 1985, and 2000 within the classes.
	
	Also, in [16] studied the 20 years spatiotemporal LULC in Hawalbagh block India, the supervised classification using, Maximum	Likelihood Classification was used. The result shows 5 LULC	classes namely agriculture, barren, built-up, vegetation, and water	body, where 3.51% and 3.55% increase in vegetation and built-up	areas, while a decrease of 5.46%, 1.52% and 0.08% of barren land	agriculture, and water body respectively was noticed.\\
	
	Furthermore, in [17] study the LULC change of watershed in Pakistan from 1992 to 2012 using the supervised classification of	maximum likelihood algorithm in Erdas Imagine. The finding	shows 5 LULC classes agriculture, bare soil/rocks, settlements,	vegetation and water. Also, the water body and vegetation are	decreasing in favour of settlements, agriculture and bare soil rapidly from 38.2% and 74.3% respectively.\\
	
	Also, in [18] study, both unsupervised (ISODATA) and supervised (MLA) methods were used for LULC classification. Change
	detection and Markov change analysis methods used to measure 	the LULC changes and generate future LULC map respectively in 	Mansoura and Talkha of Egypt from 1985 to 2010. The finding 	shows four LULC classes viz agriculture, barren land, built-up	area and water body. Also, a significant change was noticed in	agricultural land and built-up area to tune of 33$\%$ decrease and	30$\%$ increase respectively, while barren land and water bodies
	changes were minimal.\\
	
	Similarly, in [19] studied the LULC classification of Sawantwadi taluka, in India. The hybrid, parametric (MLA and ISODATA),	and nonparametric (DT) methods were used. The finding shows	the classified LULC of the forest, water, built-up, agriculture, 	plantation, fallow land, open and dense shrubland, stone quarry,and grassland with an accuracy assessment of 93$\%$ and koppa of	0.92.\\
	
	Also, in [20] measured the LULC change in Seramban. In the	study, Natural Breaks (Jenks) and Normalized Difference Vegetation Index (NDVI) methods were used for classification and difference from 1990 to 2000. The result shows four classes of	LULC viz barren land, built-up area, vegetation and water body. A	13$\%$ decrease in vegetation cover was noticed while other land	use/ land cover increase by 3.7$\%$ accordingly with an accuracy	assessment of 87$\%$ and 88$\%$ respectively.\\
	
	Likewise, in [21] studied twenty-five years the spatiotemporal	urban growth of Kuala Lumpur, using the Maximum Likelihood	Classification (MLC) method for years 1989, 2001 and 2014. The	result shows 4 LULC classes agriculture, urban/built-up, forest,	and water body. Also, a rapid increase of the built-up areas and	agricultural land was noticed while other land covers decrease very significantly.\\
	
	Also, in [22] used NDVI method to study the LULC change of	Sambas watershed, in Malaysia for the years 1990, 2002 and 2013.	The results show 5 LULC classes viz barren land, forest, grassland,	shrub and water body. A significant decrease in the forest cover was noticed, while barren land and grassland was increasing accordingly throughout the period.\\
	
	Also, in [23] studied the ten years LULC changes of Aluva taluk, in India from 2000 to 2010. Supervised classification (MLA) and	Change Detection Analysis were used for LULC classification and	mapping. The result shows 8 LULC classes viz Agriculture, Built	up, Cropland, Fallow land, Forest deciduous, Forest evergreen,	Plantation, and Waterbody. A significant change in some LULC	was noticed.	Furthermore, in [24] used change detection matrix in the study of
	LULC change of Kolong River basin of India in the years 1967-68 and 2014. The finding shows six LULC classes viz agricultural	land, built- up, forest, open space, shrub and wetland. A significant change in two primary land use, agricultural land and built-up	area with the former decreasing in the year 1967-68 and the later
	increasing much in the year 2014 respectively.\\
	
	Similarly, in [25] used supervised Maximum likelihood classification (MCL), and multi-layer Perceptron-Markov chain analysis	(MLP-MCA) to monitor the LULC changes as well predict future	LULC changes in Patna India. The result shows seven classes of	LULC viz agriculture, built-up, Fallow land, Riverbed, Shrubs,	Vegetation, Water bodies. A slight change in the LULC was noticed across the classes, in a decrease and increases pattern. Also,	the prediction shows significant changes in the built-up area.\\
	Moreover, in [26] studied 31 years LULC change in Beressa Watershed Ethiopia from 1984 to 2015. Unsupervised ISODATA
	using Erdas imagine and Change Detection methods were used in
	LULC classification and change magnitude respectively. The finding shows six classes of LULC viz barren land, farmland, forest/plantation, grazing land, settlement, and water body. A continuous increase in settlement and farmland, while grazing land and	barren were decreasing over the three decades.\\
	
	Furthermore, in [27] studied LULC changes in Udhaim river basin	in Iraq using Landsat TM image for 2006 and OLI 2015. Spectral	indices (NDVI, NDBI, NDWI, NDBaI, and CI) methods were	used to study for LULC classification and changes. The finding	shows 5 LULC according to each index viz bare land, built-up,	soil crust, vegetation and water body. Also, significant changes	were noticed in the LULC with 3% increase soil crust, and 2.43%,	0.6%, 0.55% and 0.22% decrease in vegetation cover, built-up	area, bare land and water body respectively.\\
	
	Likewise, in [28] study the LULC of Kan basin from 2000 to 2016.	Supervised classification of (MLA) method and Change Detection	Analysis were used. The finding shows 5 LULC classes viz bareland, built up, garden, pasture, and water body. Also, found a	slight increase of 0.3% and 0.2 % of pasture and build-up areas	respectively, while bare land, garden, and water body decrease	slightly over sixteen years by 0.4% and 0.01% respectively with	an accuracy assessment of 86% and 89% for years 2000 and 2016	respectively.\\
	
	Moreover, in [29] study the LULC change of hotspot area in Pune	region using Landsat images for 1972, 1992, and 2012. Change	Detection and Statistical Cluster Analysis method for LULC	change were used. The result shows 10 LULC classes viz cropland,	fallow land, forests, industrial, rivers, rural, tree clads and Wastelands with an accuracy assessment range from 77% to 97%. A	significant change was noticed, an increase in fallow land, industrial, and built-up areas around the hotspot region.\\
	
	Also, in [30] studied ten years LULC change and transformations
	in Kanyakumari coast India. The study used supervised classification of (MLA) and Change Detection. The finding shows 8 LULC
	classes viz barren land, built-up, beach face, cultivable lands, fallow land, mining, vegetation, and water body. A significant	change in the coastal LULC of Kanyakumari was noticed, some	LULC changing to another over the ten-year period with accuracy	assessment of 81.16% and 77.52% for image 2000 and 2011 respectively.\\
	
	Also, in [31] study the LULC change in Tanguar Haor, Bangladesh. The study used supervised classification in (MLA) for classification and CVA, NDVI and NDWI analysis were used to for	change detection analysis. The result shows 4 LULC classes deep water, vegetation, shallow water, and settlement. A significant change in the LULC with about 40% of the total area transformation, i.e. changing from one LULC to another.\\
	
	Furthermore, in [32] used Google Earth and GIS Operation to	study the LULC changes in Muar sub-district in Malaysia from
	the year 2010 to 2015. The results show 6 LULC classes viz agriculture, barren land, built-up, forest, open/ reaction space and	roads. Also shows a significant change in the overall LULC across	the district, i.e. some land covers a been converted into another	form.\\
	
	Also, in [33] studied the spatial-temporal LULC change in Astrakhan city, Russia, from the year 2000 to 2015. Supervised (MLA)	and change detection analysis was used to classify the LULC and	monitor the LULC change within the period. The result shows 5	LULC classes viz agriculture, bare-land, settlements, vegetation	and water body. It further shows large vegetation dilapidation and	water logging in different parts of the. Astrakhan city	Also, in [34] used supervised (MLA) classification and Stochastic	Markov (St Markov) method, to study the LULC change and predict future urban land use in Jodhpur City in India from 1990 to	2000. The finding shows 5 LULC classes viz built-up, mining area,	other land, vegetation, and water body. It further shows the rates	of changes from one LULC to the other.\\
	
	Furthermore, in [35] studied the LULC change Khan-Kali watershed and Anas River from Gujarat, India between the year 2001	and 2011. Supervised classification (MLA) and NDVI and NDWI	methods were used for LULC classification and change detection	respectively. The results 7 LULC classes viz agriculture, barren	land, built-up, forest, riverine sand, shrubland, and water body	with an overall accuracy assessment of 91.8% and 95.5% for years	2001 and 2011 respectively. Also, a significant increase in the	water body and shrub land while a decrease in the forest, barren
	land and riverine LULC.\\
	
	Moreover, in [36] studied LULC classification in Okara, Pakistan.	The supervised classification of MLA and Synthetic Aperture	Radar (SAR) methods were used. The finding shows four classes	of LULC, barren land, built-up, water body and vegetation, with	an overall accuracy of 80 % and 0.69 Kappa coefficients.	Also, in [37] in their study, Google Earth Engine (GEE) and Normalized Difference Vegetation Index (NDVI) method were used	for LULC classification and detect major LULC changes from	1985 to 2014 in Beijing respectively. The finding shows seven	classes of LULC viz cropland, grassland, forest, shrub, water body,	built-up, and barren land, with an overall accuracy of 86.61%.\\
	
	However, R proved to be a quick and efficient tool for time series analysis and map classification in the study.
	3. Conclusion
	The main objective of this article is to review the previous studies	of the spatiotemporal LULC changes using RS and GIS. It was	observed in the review; different GIS software is used for different	methods by many researchers such as supervised MLA, MLC,	CD, GEE, SAR, NDVI, MCA, DT, the Hybrid, Transition matrix,	Corine land cover nomenclature, and the unsupervised classification. Supervised classification using (MLA) is the predominant	method used for LULC classification, while other methods were	either used to detect the LULC changes or future LULC prediction. Also, the classified classes are a mostly built-up area or urban areas, agriculture, water body, barren land, forest, shrubland,
	and other land. Furthermore, RS and GIS serve as best and efficient tools for general and more detailed LULC change analysis.\\
	
	In conclusion, the study of LULC changes provides detail on the	type of changes and magnitude in a spatial location, his help the	government, and general land user a guide for proper environmental monitoring and management.
	

	[6] dos Santos, J. C. N., de Andrade, E. M., Medeiros, P. H. A.,	Guerreiro, M. J. S., and Palácio, H. A. Q. (2017). Land use impact on	soil erosion at different scales in the Brazilian semi-arid. Revista	Ciencia Agronomica, 48(2), 251–260.
	[7] Kamarudin, M. K. A., Abd. Wahab, N. A., Mamat, A. F., Juahir, H.,	Toriman, M .E., Wan, N. F. N., Ata, F. M., Ghazali, A., Anuar, A.,	and Saad, M. H. M. (2018). Evaluation of annual sediment load	production in Kenyir Lake reservoir, Malaysia. International	Journal of Engineering and Technology, 7(3.14 Special Issue 14),	pp. 55-60.
	[8] Prestele, R., Arneth, A., Bondeau, A., De Noblet-Ducoudré, N.,	Pugh, T. A. M., Sitch, S., Stehfest. E., and Verburg, P. H. (2017).	Current challenges of implementing anthropogenic land-use and	land-cover change in models contributing to climate change
	assessments. Earth System Dynamics, 8(2), 369–386
	[9] Abd El-Kawy, O. R., Rød, J. K., Ismail, H. A., and Suliman, A. S.	(2011). Land use and land cover change detection in the western	Nile delta of Egypt using remote sensing data. Applied Geography,	31(2), 483–494.
	[10] Fichera, C. R., Modica, G., and Pollino, M. (2012). Land cover	classification and change-detection analysis using multi-temporal	remote sensed imagery and landscape metrics. European Journal of	Remote Sensing, 45(1), 1–18.
	[11] Dewan, A. M., Yamaguchi, Y., and Rahman, M. Z. (2012).	Dynamics of land use/cover changes and the analysis of landscape	fragmentation in Dhaka Metropolitan, Bangladesh. GeoJournal,	77(3), 315–330.
	[12] Mallupattu, P. K., Reddy, J., and Reddy, S. (2013). Analysis of land	use / land cover changes using remote sensing data and GIS at an	Urban Area, Tirupati, India. Scientific World Journal, 2013, 1–7.
	[13] Ikiel, C., Ustaoglu, B., Dutucu, A. A., and Kilic, D. E. (2013).	Remote sensing and GIS-based integrated analysis of land cover	change in Duzce plain and its surroundings (north western Turkey).	Environmental Monitoring andAssessment, 185(2), 1699–1709.
	[14] Were, K. O., Dick, T. B., and Singh, B. R. (2013). Remotely sensing	the spatial and temporal land cover changes in Eastern Mau forest	reserve and Lake Nakuru drainage basin, Kenya. Applied	Geography, 41, 75–86.
	
	[15] Mir, S. I., Muhammad Barzani, G., Mohd Ekhwan, T., Sahibin, A.,	and Zularisam, A. W. (2012). Application of GIS for detecting	changes of land use and land cover in Tasik Chini watershed,	Pahang, Malaysia. International Journal of Civil Engineering and	Geo-Environmental, 3, 13-21
	
	[16] Rawat, J. S. e, and Kumar, M. (2015). Monitoring land use/cover	change using remote sensing and GIS techniques: A case study of	Hawalbagh block, district Almora, Uttarakhand, India. The	Egyptian Journal of Remote Sensing and Space Science, 18(1), 77–
	84.
	[17] Butt, A., Shabbir, R., Ahmad, S. S., and Aziz, N. (2015). Land use	change mapping and analysis using Remote Sensing and GIS: A	case study of Simly watershed, Islamabad, Pakistan. The Egyptian	Journal of Remote Sensing and Space Science, 18(2), 251–259.
	[18] Hegazy, I. R., and Kaloop, M. R. (2015). Monitoring urban growth	and land use change detection with GIS and remote sensing	techniques in Daqahlia governorate Egypt. International Journal of	Sustainable Built Environment, 4(1), 117-124.
	[19] Kantakumar, L. N., and Neelamsetti, P. (2015). Multi-temporal land	use classification using hybrid approach. Egyptian Journal of	Remote Sensing and Space Science, 18(2), 289–295.
	[20] Aburas, M. M., Abdullah, S. H., Ramli, M. F., and Ash’aari, Z. H.	(2015). Measuring land cover change in Seremban, Malaysia using	NDVI index. Procedia Environmental Sciences, 30, 238–243.
	
\end{document}