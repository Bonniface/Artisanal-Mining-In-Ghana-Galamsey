% Chapter 2

\chapter{CHAPTER TWO\\2.0 LITERATURE REVIEW} % Main chapter title

\label{Chapter2} % For referencing the chapter elsewhere, use \ref{Chapter1} 

%----------------------------------------------------------------------------------------

% Define some commands to keep the formatting separated from the content 
%\newcommand{\keyword}[1]{\textbf{#1}}
%\newcommand{\tabhead}[1]{\textbf{#1}}
%\newcommand{\code}[1]{\texttt{#1}}
%\newcommand{\file}[1]{\texttt{\bfseries#1}}
%\newcommand{\option}[1]{\texttt{\itshape#1}}

%----------------------------------------------------------------------------------------
\section{Introduction}
Peralta \textit{et al.,} formulate a susceptible-vaccinated-infected-recovered (SVIR) model by incorporating the vaccination of newborns, vaccine age, and mortality induced by the disease into the SIR epidemic model. It is assumed that the period of immunity induced by vaccines varies depending on the vaccine-age. They performed a nonlinear stability analysis, by means of the Lyapunov function techniques and LaSalle's Invariance Principle for semiflows. They showed that the classical threshold condition for the effective reproductive number, $ R_{v} $, holds: $ R_{v} > 1 $; then the endemic steady $ E^{*} $ is globally asymptotically stable, whereas if $ R_{v} \leq 1$, then the infection-free steady state $ E_{0} $ is globally asymptotically stable. \parencite{peralta2015global}.

%----------------------------------------------------------------------------------------