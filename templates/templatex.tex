\documentclass[12pt,a4paper]{book}
\usepackage[utf8]{inputenc}
\usepackage[T1]{fontenc}
\usepackage[english]{babel}
\usepackage{amsmath}
\usepackage{amsfonts}
\usepackage{amssymb}
\usepackage{mathptmx}
\usepackage{mathpazo}
\usepackage{aligned-overset}
\usepackage{mathrsfs}
\usepackage{makeidx}
\usepackage{times}
\usepackage{graphicx}
\usepackage[left=3.75cm, right=2.5cm, top=2.5cm, bottom=2.5cm]{geometry}
\usepackage[section]{placeins}
\newtheorem{theorem}{Theorem}[section]
\newtheorem{dfn}{Definition}[section]
\newtheorem{note}{Note}[section]
\usepackage{titlesec}
\usepackage[style = apa, backend = biber, natbib = true]{biblatex}
\addbibresource{references.bib}
\usepackage{fancyhdr}
\pagestyle{fancy} % Turn on the style
\fancyhf{} % Start with clearing everything in the header and footer
% Set the right side of the footer to be the page number
\fancyfoot[R]{\thepage}

% Redefine plain style, which is used for titlepage and chapter beginnings
% From https://tex.stackexchange.com/a/30230/828
\fancypagestyle{plain}{%
	\renewcommand{\headrulewidth}{0pt}%
	\fancyhf{}%
	\fancyfoot[R]{\thepage}%
}
% \usepackage{xcolor}
\makeatletter
%%%%% ADDING LATEX PACKAGES
% add hyperref package with options from YAML %
\usepackage[pdfpagelabels]{hyperref}
% handle long urls
\usepackage{xurl}
% change the default coloring of links to something sensible
\usepackage{xcolor}

$if(linkcolor-rgb)$
\definecolor{mylinkcolor}{RGB}{$linkcolor-rgb$}
$endif$
$if(urlcolor-rgb)$
\definecolor{myurlcolor}{RGB}{$urlcolor-rgb$}
$endif$
$if(citecolor-rgb)$
\definecolor{mycitecolor}{RGB}{$citecolor-rgb$}
$endif$

$if(colored-not-bordered-links)$
\hypersetup{
  hidelinks,
  colorlinks,
  linktocpage=$if(toc-link-page-numbers)$$toc-link-page-numbers$$else$false$endif$,
  linkcolor=$if(linkcolor-rgb)$mylinkcolor$else$.$endif$,
  urlcolor=$if(urlcolor-rgb)$myurlcolor$else$.$endif$,
  citecolor=$if(citecolor-rgb)$mycitecolor$else$.$endif$
}

$else$
\hypersetup{
  colorlinks=false,
  linktocpage=$if(link-toc-page)$$link-toc-page$$else$true$endif$,
  linkbordercolor=$if(linkcolor-rgb)$mylinkcolor$else$.$endif$,
  urlbordercolor=$if(urlcolor-rgb)$myurlcolor$else$.$endif$,
  citebordercolor=$if(citecolor-rgb)$mycitecolor$else$.$endif$
}
$endif$


% add float package to allow manual control of figure positioning %
\usepackage{float}

% enable strikethrough
\usepackage[normalem]{ulem}

% use soul package for correction highlighting
\usepackage{color, soulutf8}
\definecolor{correctioncolor}{HTML}{CCCCFF}
\sethlcolor{correctioncolor}
\newcommand{\ctext}[3][RGB]{%
  \begingroup
  \definecolor{hlcolor}{#1}{#2}\sethlcolor{hlcolor}%
  \hl{#3}%
  \endgroup
}
\soulregister\ref7
\soulregister\cite7
\soulregister\citet7
\soulregister\autocite7
\soulregister\textcite7
\soulregister\pageref7

%%%%% FIXING / ADDING THINGS THAT'S SPECIAL TO R MARKDOWN'S USE OF LATEX TEMPLATES
% pandoc puts lists in 'tightlist' command when no space between bullet points in Rmd file,
% so we add this command to the template
\providecommand{\tightlist}{%
  \setlength{\itemsep}{0pt}\setlength{\parskip}{0pt}}
 
% UL 1 Dec 2018, fix to include code in shaded environments
$if(highlighting-macros)$
$highlighting-macros$

%UL set white space before and after code blocks
\renewenvironment{Shaded}
{
  \vspace{$space-before-code-block$}%
  \begin{snugshade}%
}{%
  \end{snugshade}%
  \vspace{$space-after-code-block$}%
}
$endif$

% User-included things with header_includes or in_header will appear here
% kableExtra packages will appear here if you use library(kableExtra)
$for(header-includes)$
$header-includes$
$endfor$


%UL set section header spacing
\usepackage{titlesec}
% 
\titlespacing\subsubsection{0pt}{24pt plus 4pt minus 2pt}{0pt plus 2pt minus 2pt}


%UL set whitespace around verbatim environments
\usepackage{etoolbox}
\makeatletter
\preto{\@verbatim}{\topsep=0pt \partopsep=0pt }
\makeatother


%%%%%%% PAGE HEADERS AND FOOTERS %%%%%%%%%
\usepackage{fancyhdr}
\setlength{\headheight}{15pt}
\fancyhf{} % clear the header and footers
\pagestyle{fancy}
\renewcommand{\chaptermark}[1]{\markboth{\thechapter. #1}{\thechapter. #1}}
\renewcommand{\sectionmark}[1]{\markright{\thesection. #1}} 
\renewcommand{\headrulewidth}{0pt}

$if(running-header)$
\fancy$running-header-foot-or-head$[$running-header-position-leftmark$]{\emph{\leftmark}} 
\fancy$running-header-foot-or-head$[$running-header-position-rightmark$]{\emph{\rightmark}} 
$endif$

% % UL page number position 
% \fancy$ordinary-page-number-foot-or-head$[$ordinary-page-number-position$]{\emph{\thepage}} %regular pages
% \fancypagestyle{plain}{\fancyhf{}\fancy$chapter-page-number-foot-or-head$[$chapter-page-number-position$]{\emph{\thepage}}} %chapter pages


%%%%% SELECT YOUR DRAFT OPTIONS
% This adds a "DRAFT" footer to every normal page.  (The first page of each chapter is not a "normal" page.)
$if(draft-mark)$
\fancy$draft-mark-foot-or-head$[$draft-mark-position$]{\emph{DRAFT Printed on \today}}
$endif$

% IP feb 2021: option to include line numbers in PDF
$if(includeline-num)$
\usepackage{lineno}
\linenumbers
$endif$

% for line wrapping in code blocks
$if(line-wrapping-in-code)$
\usepackage{fancyvrb}
\usepackage{fvextra}
\DefineVerbatimEnvironment{Highlighting}{Verbatim}{breaklines=true, breakanywhere=true, commandchars=\\\{\}}
$endif$

% This highlights (in blue) corrections marked with (for words) \mccorrect{blah} or (for whole
% paragraphs) \begin{mccorrection} . . . \end{mccorrection}.  This can be useful for sending a PDF of
% your corrected thesis to your examiners for review.  Turn it off, and the blue disappears.
$if(corrections)$
\correctionstrue
$endif$


%%%%% BIBLIOGRAPHY SETUP
% Note that your bibliography will require some tweaking depending on your department, preferred format, etc.
% If you've not used LaTeX before, I recommend reading a little about biblatex/biber and getting started with it.
% If you're already a LaTeX pro and are used to natbib or something, modify as necessary.
% Either way, you'll have to choose and configure an appropriate bibliography format...

% this enables pandoc citations
$if(csl-refs)$
\newlength{\cslhangindent}
\setlength{\cslhangindent}{1.5em}
\newlength{\csllabelwidth}
\setlength{\csllabelwidth}{3em}
\newlength{\cslentryspacingunit} % times entry-spacing
\setlength{\cslentryspacingunit}{\parskip}
\newenvironment{CSLReferences}[2] % #1 hanging-ident, #2 entry spacing
 {% don't indent paragraphs
  \setlength{\parindent}{0pt}
  % turn on hanging indent if param 1 is 1
  \ifodd #1
  \let\oldpar\par
  \def\par{\hangindent=\cslhangindent\oldpar}
  \fi
  % set entry spacing
  \setlength{\parskip}{$refs-space-between-entries$}
  \setlength{\baselineskip}{$refs-line-spacing$}
 }%
 {}
\usepackage{calc}
\newcommand{\CSLBlock}[1]{#1\hfill\break}
\newcommand{\CSLLeftMargin}[1]{\parbox[t]{\csllabelwidth}{#1}}
\newcommand{\CSLRightInline}[1]{\parbox[t]{\linewidth - \csllabelwidth}{#1}\break}
\newcommand{\CSLIndent}[1]{\hspace{\cslhangindent}#1}
$endif$

$if(use-biblatex)$
\usepackage[$bib-latex-options$]{biblatex}

$for(bibliography)$
\addbibresource{$bibliography$}
$endfor$

% This makes the bibliography left-aligned (not 'justified') and slightly smaller font.
\renewcommand*{\bibfont}{\raggedright\small}

$endif$

$if(use-natbib)$
\usepackage{natbib}
\setcitestyle{$natbib-citation-style$}
\bibliographystyle{$natbib-bibliography-style$}
\addto\captionsenglish{%
  \renewcommand{\bibname}{}
  \renewcommand{\bibsection}{}
}

% This makes the bibliography left-aligned (not 'justified') and slightly smaller font.
\renewcommand*{\bibfont}{\raggedright\small}

$endif$



\makeatother
\titleformat{\section}{\bfseries\normalfont\centering\bf}{\thesection}{0em}{}
\begin{document}
	\pagestyle{plain}
	\openup 1 em
	\begin{titlepage}
		\pagenumbering{roman}
		%\addcontentsline{toc}{section}{Title Page}
		\begin{center}
			
			\begin{figure}[h]
				\begin{center}
				$if(university-logo)$
          \def\crest{{\includegraphics[width=$university-logo-width$]{$university-logo$}}}
            $else$
            \def\crest{}
            $endif$
					
				\end{center}
			\end{figure}
			
			{\normalfont{\textbf{UNIVERSITY OF ENERGY AND NATURAL RESOURCES, SUNYANI}}}\\
			\vspace{1.7cm}
			{\normalfont \title{$title$}}
			
			
			\vspace{2.7cm}
			
			{\normalfont {\author{$author$}}}
			\vspace{1.7cm}
			
			
			\begin{center}
				{\normalfont \textbf{DEPARTMENT OF MATHEMATICS AND STATISTIC}\\
					\textbf{SCHOOL OF SCIENCE}}\\
				\vspace{5.3cm}
				
				
				{\normalfont \date{$date$}}
			\end{center}
			
		\end{center}
		
		\vfill
	\end{titlepage}
	\pagenumbering{roman}
	\begin{titlepage}
		\begin{center}
			{\normalfont \title{$title$}}
			
			\vspace{3.7cm}
			{\normalfont {by}}\\
			\vspace{1cm}
			{\normalfont {Kalong Boniface - UEB3603118\\ B.Sc. Actuarial Science Location}}
			\vspace{1.7cm}
			
			
			\begin{center}
				{\normalfont A Thesis submitted to the Department of Mathematics and Statistics, School of Science, University of Energy and Natural Resources, Sunyani in partial fulfillment of the requirements for the degree of Bachelor of Science in Mathematics }\\
				\vspace{2cm}	
				{\normalfont \date{$date$}}
			\end{center}
			
		\end{center}
		
		\vfill
	\end{titlepage}	
	\newpage
	{\section*{DECLARATION AND CERTIFICATION}
	\addcontentsline{toc}{section}{CERTIFICATION}
	personal declaration here...
	\\
		Candidate's Signature:	................................. \hspace\fill 
		Date: .................................	 \\
	\begin{center}\textbf{\normalfont \textbf{Supervisor's Certification}}\end{center}
	This study was carried out under the supervisory committee of (Names of all Supervisors) in accordance with the guidelines on supervisions of graduate studies.\\\\
	Major Supervisor's Name and Qualifications \\
	\\
	Signature:	................................. \hspace\fill 
	Date: .................................	 \\
	\begin{center}\textbf{\normalfont \textbf{Co-Supervisor's Certification}}\end{center}
	Co-Supervisor's Name and Qualifications\\
	\\
	Signature:	................................. \hspace\fill 
	Date: .................................	 \\
	\newline\\
	Co-Supervisor's Name and Qualifications\\
	\\
	Signature:	................................. \hspace\fill 
	Date: .................................	 \\
	\newpage
	\section*{\textbf{ABSTRACT}}
	\addcontentsline{toc}{section}{ABSTRACT}
	Write your abstract here 
	\newpage	
	\begin{center}\section*{DEDICATION}\end{center}
	\addcontentsline{toc}{section}{DEDICATION}
	
	Write dedication here   
	\newpage	
	\begin{center}\section*{ACKNOWLEDGMENTS}\end{center}
	\addcontentsline{toc}{section}{ACKNOWLEDGMENTS}
	
	Write Acknowledgment here
	
	\newpage
	

%%%%% CHOOSE YOUR SECTION NUMBERING DEPTH HERE
% You have two choices.  First, how far down are sections numbered?  (Below that, they're named but
% don't get numbers.)  Second, what level of section appears in the table of contents?  These don't have
% to match: you can have numbered sections that don't show up in the ToC, or unnumbered sections that
% do.  Throughout, 0 = chapter; 1 = section; 2 = subsection; 3 = subsubsection, 4 = paragraph...

% The level that gets a number:
\setcounter{secnumdepth}{$section-numbering-depth$}
% The level that shows up in the ToC:
\setcounter{tocdepth}{$toc-depth$}



%%%%% MINI TABLES
% This lays the groundwork for per-chapter, mini tables of contents.  Comment the following line
% (and remove \minitoc from the chapter files) if you don't want this.  Un-comment either of the
% next two lines if you want a per-chapter list of figures or tables.
$if(mini-toc)$
  \dominitoc % include a mini table of contents
$endif$
$if(mini-lof)$
  \dominilof  % include a mini list of figures
$endif$
$if(mini-lot)$
  \dominilot  % include a mini list of tables
$endif$

% This aligns the bottom of the text of each page.  It generally makes things look better.
\flushbottom

% This is where the whole-document ToC appears:
\tableofcontents

$if(lof)$
\listoffigures
	\mtcaddchapter
  	% \mtcaddchapter is needed when adding a non-chapter (but chapter-like) entity to avoid confusing minitoc
$endif$

% Uncomment to generate a list of tables:
$if(lot)$
\listoftables
  \mtcaddchapter


%%%%% CHAPTERS
% Add or remove any chapters you'd like here, by file name (excluding '.tex'):
\flushbottom

% all your chapters and appendices will appear here
$body$

%%%%% REFERENCES
$if(use-biblatex)$
\setlength{\baselineskip}{0pt} % JEM: Single-space References

% we are setting the title for the references section in front-and-back-matter/99-references_heading.Rmd
{\renewcommand*\MakeUppercase[1]{#1}%
\printbibliography[heading=none]}

$endif$

$if(use-natbib)$
\bibliography{$for(bibliography)$$bibliography$$sep$,$endfor$}
$endif$

\end{document}