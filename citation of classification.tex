 Aburas, M. M., Abdullah, S. H., Ramli, M. F., and Ash’aari, Z. H.
(2015). Measuring land cover change in Seremban, Malaysia using
NDVI index. Procedia Environmental Sciences, 30, 238–243.
[21] Boori, M. S., Netzband, M., Choudhary, K., and Voženílek, V.
(2015). Monitoring and modeling of urban sprawl through remote
sensing and GIS in Kuala Lumpur, Malaysia. Ecological Processes,
4(1), 1-10.

Kantakumar, L. N., and Neelamsetti, P. (2015). Multi-temporal land
use classification using hybrid approach. Egyptian Journal of
Remote Sensing and Space Science, 18(2), 289–295.

 Hegazy, I. R., and Kaloop, M. R. (2015). Monitoring urban growth
and land use change detection with GIS and remote sensing
techniques in Daqahlia governorate Egypt. International Journal of
Sustainable Built Environment, 4(1), 117-124.

 Butt, A., Shabbir, R., Ahmad, S. S., and Aziz, N. (2015). Land use
change mapping and analysis using Remote Sensing and GIS: A
case study of Simly watershed, Islamabad, Pakistan. The Egyptian
Journal of Remote Sensing and Space Science, 18(2), 251–259.


 Ikiel, C., Ustaoglu, B., Dutucu, A. A., and Kilic, D. E. (2013).
Remote sensing and GIS-based integrated analysis of land cover
change in Duzce plain and its surroundings (north western Turkey).
Environmental Monitoring and Assessment, 185(2), 1699–1709.

 Fichera, C. R., Modica, G., and Pollino, M. (2012). Land cover
classification and change-detection analysis using multi-temporal
remote sensed imagery and landscape metrics. European Journal of
Remote Sensing, 45(1), 1–18.


 Kamarudin, M. K. A., Abd. Wahab, N. A., Mamat, A. F., Juahir, H.,
Toriman, M .E., Wan, N. F. N., Ata, F. M., Ghazali, A., Anuar, A.,
and Saad, M. H. M. (2018). Evaluation of annual sediment load
production in Kenyir Lake reservoir, Malaysia. International
Journal of Engineering and Technology, 7(3.14 Special Issue 14),
pp. 55-60.


 dos Santos, J. C. N., de Andrade, E. M., Medeiros, P. H. A.,
Guerreiro, M. J. S., and Palácio, H. A. Q. (2017). Land use impact on
soil erosion at different scales in the Brazilian semi-arid. Revista
Ciencia Agronomica, 48(2), 251–260.