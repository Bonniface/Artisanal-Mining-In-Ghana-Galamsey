% Chapter 1

\chapter{CHAPTER ONE\\1.0 INTRODUCTION} % Main chapter title
\section{Introduction}
\label{Chapter1} % For referencing the chapter elsewhere, use \ref{Chapter1} 

%-------------------------------------------------------------------------------------

One would anticipate that the majority of emerging nations, which are still in the early stages of economic development and growth, would have a high forest cover and little deforestation. This, however, has not been the case. Ghana is a lower-middle-income nation that is still working toward middle-income classification. However, it has already begun to see a deforestation rate that is comparable to that of middle-income countries. The rapid population expansion, clearing of field for Galamsey operation,increased domestic need of wood for things like fuel, furniture, construction, and timber exports have all contributed to this trend, Bush fires in the 1980s, climate change, and lax law enforcement have all had an impact.

The purpose of this paper is to establish an understanding in time series analysis on remotely sensed data. Which will introduced us to the fundamentals of time series modeling, including decomposition, autocorrelation and modeling historical changes in Galamsey Operation in Ghana, the Cause,Dangers and it's Environmental impact.

Galamsey also known as "gather them and sell",\parencite{Mantey2017} is the term given by local Ghanaian for illegal small-scale gold mining in Ghana . The major cause of Galamsey is unemployment among the youth in Ghana \parencite{Gracia2018}. Young university graduates rarely find work and when they do it hardly sustains them. The result is that these youth go the extra mile to earn a living for themselves and their family.

Another factor is that lack of job security. On November 13, 2009 a collapse occurred in an illegal, privately owned mine in Dompoase, in the Ashanti Region of Ghana. At least 18 workers were killed, including 13 women, who worked as porters for the miners. Officials described the disaster as the worst mine collapse in Ghanaian history \parencite{BBCNews2009}.

Illegal mining causes damage to the land and water supply \parencite{Ansah2017} . In March 2017, the Minister of Lands and Natural Resources, Mr. John Peter Amewu, gave the Galamsey operators/illegal miners a three-week ultimatum to stop their activities or be prepared to face the law \parencite{Allotey2017} . The activities by Galamseyers have depleted Ghana's forest cover and they have caused water pollution, due to the crude and unregulated nature of the mining process \parencite{Gyekye}.

Under current Ghanaian constitution, it is illegal to operate as galamseyer.That is to dig on land granted to mining companies as concessions or licenses and any other land in search for gold. In some cases, Galamseyers are the first to discover and work extensive gold deposits before mining companies find out and take over. Galamseyers are the main indicator of the presence of gold in free metallic dust form or they process oxide or sulfide gold ore using liquid mercury.

Between 20,000 to 50,000, including thousands from China are believed to be engaged in Galamsey in Ghana.But according to the Information Minister 200,000 and nearly 3 million people, recently are now into Galamsey operation and rely on it for their livelihoods \parencite{goldgu2017}. Their operations are mostly in the southern part of Ghana where it is believe to have substantial reserves of gold deposits, usually within the area of large mining companies \parencite{Barenblitt2021} . As a group, they are economically disadvantaged. Galamsey settlements are usually poorer than neighboring agricultural villages. They have high rates of accidents and are exposed to mercury poisoning from their crude processing methods. Many women are among the workers, acting mostly as porters for the miners.

\section{Background of The Study}

As Galamsey is considered an illegal activity, they operations are hidden to the eyes of the authorities.So locating them is quite tricky ,but with satellite imagery ,it now possible to locate their operating and put an end to it. One of the features of Google Earth Engine is the ability to access years of satellite imagery without needing to download, organize, store and process this information. For instance, within the Satellite image

collection, now it possible to access imagery back to the 90's, allowing us to look at areas of interest on the map to visualize and quantify how much things has changed over time. With Earth Engine, Google maintains the data and offers it's computing power for processing.Users can now access hundreds of time series images and analyze changes across decades using GIS and R or other programming language to analyze these datasets.

\section{Problem Statement}
The Footprint of Galamsey is Spreading at a very faster rate, causing vegetation loss.Other factors accounting to vegetation loss may largely include climate change,urban and exurban development, bush fires. But not much works or research has been done to tell the extent to which Galamsey causes vegetation loss. This research attempts to segregate the variability climate is responsible for in vegetation loss so as to attribute the residual variability to Galamsey and other related activities such as bush-fires etc.

\section{Research Questions}
To address the challenge of the vegetation variability in this work, the following several statements were formed:

\begin{itemize}
	\item  Are there any changes in vegetation cause by Galamsey and Climate change in Ghana?
	\item Is there any relationship between vegetation loss and Climate change in Ghana?
\end{itemize}

\section{Research Objectives}
The purpose is to establish an understanding of time series analysis on remotely sensed data. We will be introduced to the fundamentals of time series modeling, including decomposition, auto-correlation, and modeling historical changes.Unfortunately, the causes of deforestation and forest degradation have not been adequately defined in the environmental literature. According to \paragraph{hosonuma2012assessment}, there are four causes of forest degradation: logging for wood, uncontrolled fires, livestock grazing in forests, and
five deforestation drivers, and fuel (wood/charcoal) (commercial agriculture, subsistence agriculture, mining, infrastructure and urban expansion). According to these drivers, this study reclassifies the drivers for an empirical examination into human conduct or activity and climatic change by;

\begin{itemize}
	\item Performing time series analysis on satellite derived vegetation indices
	
	\item Estimate the extent to which Galamsey causes vegetation loss in Ghana.
	
	\item Single out the variability climate is responsible for.
\end{itemize}

\section{Significance Of The Study}

There have been significant changes in vegetation cover in Ghana over the past 30 years, and these dynamics are related strongly to climatic factors such as temperature and other factors. In this study, we want to examine the effects of climatic change on Ghana's vegetation during these thirty years.

This study allows us to explore climatic differences and climate-related drivers. Additionally, it offers a chance to research how climatic variability affects the ecosystem and human health. By merging climatic and vegetation (EVI) data to understand the relationship between vegetation and climate change under tropical climate conditions, it closes research gaps in Ghana. This study explores historical and projected vegetation and climate data, by sector, impacts, key vulnerabilities and what adaptation measures can be taken. It also explores the overview for a general context of how climate change is affecting Ghana.

\section{Limitation Of The Study}

There have been significant changes in vegetation cover in Ghana over the past 30 years, and these dynamics are related strongly to climatic factors such as temperature and other factors. In this study, we want to examine the effects of climatic change on Ghana's vegetation during these thirty years.

This study allows us to explore climatic differences and climate-related drivers. Additionally, it offers a chance to research how climatic variability affects the ecosystem and human health. By merging climatic and vegetation data to understand the relationship between vegetation and climate change closes research gaps in Ghana. This study explores historical and projected vegetation and climatic data, by sector, impacts, key vulnerabilities and what adaptation measures can be taken. It also explores the overview for a general context of how climate change is affecting Ghana's Vegetation .